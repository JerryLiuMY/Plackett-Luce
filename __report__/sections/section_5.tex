\section{Conclusion}
In sum, we conclude that the data provide more evidence in favour of the simpler observation model without seniority rank covariate. From the model, we identify the strongest and the weakest players that match our observations from the exploratory data analysis. It is also found that if game $68$ was replayed, model one predicts that player $7$ has $17.4\%$ change of coming as the top player. The rejected model with seniority covariates has non-trivial effects on this prediction.

There are two noticeable limitations of our analysis: First, it is common that some of the players have not played directly with each other, and their relative skills are inferred indirectly from comparisons with the players they have commonly played with. This is likely to weaken our pairwise comparison. Second, we have a relatively low number of observations for several players whose skill level is more likely to be affected by the choice of prior. Better analysis can be performed if more data are available.

There are two potential extensions of our analysis: First of all, we could consider a wider class of prior models, for example, $t$-distributions with different degrees of freedom that represents different beliefs about the certainty of the probability of each outcome before seeing the data. A larger degree of freedom represents a higher degree of certainty about the probability of each random permutation $O=\left(O_{1}, \ldots, O_{m}\right)$. We need to check that results are insensitive to the choice of priors. Second of all, we could develop more efficient computation methods. Each MH simulation for the model with $\beta$ covariates took $30$ minutes to complete. Better computational methods (e.g. alternative proposal distributions such as random walk) can facilitate convergence, enable us to reduce subsample size and make this process more efficient.

% In this paper, we consider the Extended Plackett-Luce model that induces a flexible (discrete) distribution over permutations. The parameter space of this distribution is a combination of potentially high-dimensional discrete and continuous components, and this presents challenges for parameter interpretability and also posterior computation. Particular emphasis is placed on the interpretation of the parameters in terms of observable quantities, and we propose a general framework for preserving the mode of the prior predictive distribution. Posterior sampling is achieved using an effective simulation-based approach that does not require imposing restrictions on the parameter space. Working in the Bayesian framework permits a natural representation of the posterior predictive distribution, and we draw on this distribution to address the rank aggregation problem and also to identify the potential lack of model fit. The flexibility of the Extended Plackett-Luce model, along with the effectiveness of the proposed sampling scheme, is demonstrated using several simulation studies and real data examples.
%
% In such a setting, the $\lambda_{i j}$ corresponding to the two outer rows and columns will still have a larger variance under the prior, but we would get sensible estimates and could compare it to our categorical model.
